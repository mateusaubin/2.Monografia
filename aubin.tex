%=======================================================================
% Declarações iniciais identificando a classe de documento e
% selecionando alguns pacotes adicionais.
%
% As opções disponíveis (separe-as com vírgulas, sem espaço) são:
% - twoside: Formata o documento para impressão frente-e-verso
%   (o default é somente-frente)
% - english,brazilian,french,german,etc.: idiomas usados no documento.
%   Deve ser colocado por último o idioma principal.
%=======================================================================
\documentclass[twoside,english,brazilian]{UNISINOSmonografia}
\usepackage[utf8]{inputenc} % charset do texto (utf8, latin1, etc.)
\usepackage[T1]{fontenc} % encoding da fonte (afeta a sep. de sílabas)
\usepackage{graphicx} % comandos para gráficos e inclusão de figuras
\usepackage{bibentry} % para inserir refs. bib. no meio do texto


%=======================================================================
% Escolha do sistema para geração de referências bibliográficas.
%
% O default é usar o estilo unisinos.bst.  Comente a definição abaixo
% e descomente a linha seguinte para usar o estilo do ABNTeX (é
% necessário ter esse pacote instalado).
%
% A vantagem do unisinos.bst é que ele permite o uso de um arquivo .bib
% seguindo as orientações tradicionais do BibTeX (veja essas orientações
% em http://ctan.tug.org/tex-archive/biblio/bibtex/contrib/doc/btxdoc.pdf).
% Entretanto, o estilo não suporta algumas citações mais exóticas como
% apud.  Para isso, use o ABNTeX, mas esteja ciente de que muitas de
% suas referências serão incompatíveis com os estilos tradicionais do
% BibTeX como plain, alpha, ieeetr, entre outros.
%=======================================================================
\unisinosbst
%\usepackage[alf]{abntcite}


%=======================================================================
% Dados gerais sobre o trabalho.
%=======================================================================
\autor{Aubin}{Mateus Rauback}
\titulo{
Um Sistema de Gestão de Dispositivos Inteligentes 
Baseado em Protocolos de Gerência de Redes Voltado Para a 
Internet das Coisas
}
%\subtitulo{Versão \LaTeX}
\orientador[Prof.~Dr.]{Ávila}{Rafael Bohrer}
%\coorientador[Prof.~Dr.]{Lamport}{Leslie}
\local{São Leopoldo}
\ano{2013}

%% dados específicos para monografia de Graduação
\unidade{Unidade Acadêmica Graduação}
\curso{Curso de Bacharelado em Ciência da Computação}
\natureza{
Trabalho de Conclusão de Curso apresentado como requisito parcial
para a obtenção do título de Bacharel em Ciência da Computação
pela Universidade do Vale do Rio dos Sinos --- UNISINOS
}


%=======================================================================
% Palavras Chave.
%
% Deve ser fornecida para cada idioma.
%=======================================================================
\palavrachave{brazilian}{Internet das Coisas}
\palavrachave{english}{Internet of Things}

\palavrachave{brazilian}{Gerência de Redes}
\palavrachave{english}{Network Management}

\palavrachave{brazilian}{Protocolos de Rede}
\palavrachave{english}{Network Protocols}


%=======================================================================
% Início do documento.
%=======================================================================
\begin{document}
\capa
\folhaderosto


%=======================================================================
% Dedicatória (opcional).
%
% O texto é normalmente colocado na parte de baixo da página, alinhado
% à direita.  Mas a formatação é basicamente livre.  Só não se escreve
% a palavra 'dedicatória'.
%=======================================================================
%\begin{dedicatoria}
%Aos nossos pais.\\[4ex] % quebra a linha dando um espaçamento maior
%\begin{itshape} % faz o texto ficar em itálico
%If I have seen farther than others,\\
%it is because I stood on the shoulders of giants.\\
%\end{itshape}
%--- \textsc{Sir Isaac Newton} % \textsc é o "small caps"
%\end{dedicatoria}


%=======================================================================
% Agradecimentos (opcional).
%=======================================================================
%\begin{agradecimentos}
%Obrigado!
%\end{agradecimentos}


%=======================================================================
% Epígrafe (opcional).
%
% ``[...] o autor apresenta uma citação, seguida de indicação de autoria,
% relacionada com a matéria tratada no corpo do trabalho. Podem, também,
% constar epígrafes nas folhas de aberturas das seções primárias.''
%=======================================================================
%\begin{epigrafe}
%``\textit{Ninguém abre um livro sem que aprenda alguma coisa}''.\\
%(Anônimo)
%\end{epigrafe}


%=======================================================================
% Resumo em Português.
%
% A recomendação é para 150 a 500 palavras.
%=======================================================================
\begin{abstract}
Aqui vai o resumo
\end{abstract}


%=======================================================================
% Resumo em língua estrangeira (obrigatório somente para teses e
% dissertações).
%
% O idioma usado aqui deve necessariamente aparecer nos parâmetros do
% \documentclass, no início do documento.
%=======================================================================
\begin{otherlanguage}{english}
\begin{abstract}
Abstract goes here
\end{abstract}
\end{otherlanguage}


%=======================================================================
% Lista de Figuras (opcional).
%=======================================================================
\listoffigures


%=======================================================================
% Lista de Tabelas (opcional).
%=======================================================================
\listoftables


%=======================================================================
% Lista de Abreviaturas (opcional).
%
% Deve ser passada como parâmetro a maior das abreviaturas utilizadas.
%=======================================================================
%\begin{listadeabreviaturas}{seg., segs.}
%\item[ampl.] ampliado, -a
%\item[atual.] atualizado, -a
%\item[coord.] coordenador
%\item[N.~T.] Novo Testamento
%\item[seg., segs.] seguinte, -s
%\end{listadeabreviaturas}


%=======================================================================
% Lista de Siglas (opcional).
%
% Deve ser passada como parâmetro a maior das siglas utilizadas.
%=======================================================================
%\begin{listadesiglas}{FAPERGS}
%\item[ABNT] Associação Brasileira de Normas Técnicas
%\item[CAPES] Coordenação de Aperfeiçoamento de Pessoal de Nível Superior
%\item[FAPERGS] Fundação de Amparo à Pesquisa do Estado do Rio Grande do Sul
%\end{listadesiglas}


%=======================================================================
% Lista de Símbolos (opcional).
%
% Deve ser passado o maior (mais largo) dos símbolos utilizados.
%=======================================================================
%\begin{listadesimbolos}{Ca}
%\item[\textsuperscript{o}C] Graus Celsius
%\item[Al] Alumínio
%\item[Ca] Cálcio
%\end{listadesimbolos}


%=======================================================================
% Sumário
%=======================================================================
\tableofcontents


%=======================================================================
% Introdução
%=======================================================================
\chapter{Introdução}

% as epígrafes nos capítulos são opcionais
\epigrafecap{
	The reasonable man adapts himself to the world; 
	the unreasonable one persists in trying to adapt the world to himself. 
	Therefore all progress depends on the unreasonable man.}
{George Bernard Shaw}

Inserir introdução\ldots


\chapter{Conteúdo}

Inserir o conteúdo do trabalho\ldots

Teste \cite{Anderson95}

\section{O que?}

\emph{O que} aconteceu?

\section{Como?}

\emph{Como} foi que aconteceu?

\section{Quando}

\emph{Quando} é que aconteceu?

\section{Onde}

Mas \emph{onde} foi que aconteceu?


%=======================================================================
% Referências
%=======================================================================
\bibliography{aubin}


%=======================================================================
% Exemplo de Apêndice
% O Apêndice é utilizado para apresentar material complementar elaborado
% pelo próprio autor.  Deve seguir as mesmas regras de formatação do
% corpo principal do documento.
%=======================================================================
\appendix
\chapter{Informações Complementares}

O Apêndice é utilizado para apresentar material complementar elaborado
pelo próprio autor.  Deve seguir as mesmas regras de formatação do
corpo principal do documento.


%=======================================================================
% Exemplo de Anexo
% O Anexo é utilizado para a ``inclusão de materiais não elaborados pelo
% próprio autor, como cópias de artigos, manuais, folders, balancetes, etc.
% e não precisam estar em conformidade com o modelo''.
%=======================================================================
\annex
\chapter{Artigos Publicados}

O Anexo é utilizado para a ``inclusão de materiais não elaborados pelo
próprio autor, como cópias de artigos, manuais, folders, balancetes, etc.
e não precisam estar em conformidade com o modelo''.


\end{document}

