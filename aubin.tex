%=======================================================================
% Declarações iniciais identificando a classe de documento e
% selecionando alguns pacotes adicionais.
%
% As opções disponíveis (separe-as com vírgulas, sem espaço) são:
% - twoside: Formata o documento para impressão frente-e-verso
%   (o default é somente-frente)
% - english,brazilian,french,german,etc.: idiomas usados no documento.
%   Deve ser colocado por último o idioma principal.
%=======================================================================
\documentclass[twoside,english,brazilian]{UNISINOSmonografia}
\usepackage[utf8]{inputenc} % charset do texto (utf8, latin1, etc.)
\usepackage[T1]{fontenc} % encoding da fonte (afeta a sep. de sílabas)
\usepackage{graphicx} % comandos para gráficos e inclusão de figuras
\usepackage{bibentry} % para inserir refs. bib. no meio do texto


%=======================================================================
% Escolha do sistema para geração de referências bibliográficas.
%
% O default é usar o estilo unisinos.bst.  Comente a definição abaixo
% e descomente a linha seguinte para usar o estilo do ABNTeX (é
% necessário ter esse pacote instalado).
%
% A vantagem do unisinos.bst é que ele permite o uso de um arquivo .bib
% seguindo as orientações tradicionais do BibTeX (veja essas orientações
% em http://ctan.tug.org/tex-archive/biblio/bibtex/contrib/doc/btxdoc.pdf).
% Entretanto, o estilo não suporta algumas citações mais exóticas como
% apud.  Para isso, use o ABNTeX, mas esteja ciente de que muitas de
% suas referências serão incompatíveis com os estilos tradicionais do
% BibTeX como plain, alpha, ieeetr, entre outros.
%=======================================================================
\unisinosbst
%\usepackage[alf]{abntcite}


%=======================================================================
% Dados gerais sobre o trabalho.
%=======================================================================
\autor{Aubin}{Mateus Rauback}
\author{Mateus Rauback Aubin}
\titulo{
Um Sistema de Gestão de Dispositivos Inteligentes 
Baseado em Protocolos de Gerência de Redes Voltado Para a 
Internet das Coisas
}
%\subtitulo{Versão \LaTeX}
\orientador[Prof.~Dr.]{Ávila}{Rafael Bohrer}
%\coorientador[Prof.~Dr.]{Lamport}{Leslie}
\local{São Leopoldo}
\ano{2013}

%% dados específicos para monografia de Graduação
\unidade{Unidade Acadêmica Graduação}
\curso{Curso de Bacharelado em Ciência da Computação}
\natureza{
Trabalho de Conclusão de Curso apresentado como requisito parcial
para a obtenção do título de Bacharel em Ciência da Computação
pela Universidade do Vale do Rio dos Sinos --- UNISINOS
}


%=======================================================================
% Palavras Chave.
%
% Deve ser fornecida para cada idioma.
%=======================================================================
\palavrachave{brazilian}{Internet das Coisas}
\palavrachave{english}{Internet of Things}

\palavrachave{brazilian}{Gerência de Redes}
\palavrachave{english}{Network Management}

\palavrachave{brazilian}{Protocolos de Rede}
\palavrachave{english}{Network Protocols}


%=======================================================================
% Início do documento.
%=======================================================================
\begin{document}
\capa
\folhaderosto


%=======================================================================
% Dedicatória (opcional).
%
% O texto é normalmente colocado na parte de baixo da página, alinhado
% à direita.  Mas a formatação é basicamente livre.  Só não se escreve
% a palavra 'dedicatória'.
%=======================================================================
%\begin{dedicatoria}
%Aos nossos pais.\\[4ex] % quebra a linha dando um espaçamento maior
%\begin{itshape} % faz o texto ficar em itálico
%If I have seen farther than others,\\
%it is because I stood on the shoulders of giants.\\
%\end{itshape}
%--- \textsc{Sir Isaac Newton} % \textsc é o "small caps"
%\end{dedicatoria}


%=======================================================================
% Agradecimentos (opcional).
%=======================================================================
%\begin{agradecimentos}
%Obrigado!
%\end{agradecimentos}


%=======================================================================
% Epígrafe (opcional).
%
% ``[...] o autor apresenta uma citação, seguida de indicação de autoria,
% relacionada com a matéria tratada no corpo do trabalho. Podem, também,
% constar epígrafes nas folhas de aberturas das seções primárias.''
%=======================================================================
%\begin{epigrafe}
%``\textit{Ninguém abre um livro sem que aprenda alguma coisa}''.\\
%(Anônimo)
%\end{epigrafe}


%=======================================================================
% Resumo em Português.
%
% A recomendação é para 150 a 500 palavras.
%=======================================================================
\begin{abstract}
Aqui vai o resumo
\end{abstract}


%=======================================================================
% Resumo em língua estrangeira (obrigatório somente para teses e
% dissertações).
%
% O idioma usado aqui deve necessariamente aparecer nos parâmetros do
% \documentclass, no início do documento.
%=======================================================================
\begin{otherlanguage}{english}
\begin{abstract}
Abstract goes here
\end{abstract}
\end{otherlanguage}


%=======================================================================
% Lista de Figuras (opcional).
%=======================================================================
\listoffigures


%=======================================================================
% Lista de Tabelas (opcional).
%=======================================================================
\listoftables


%=======================================================================
% Lista de Abreviaturas (opcional).
%
% Deve ser passada como parâmetro a maior das abreviaturas utilizadas.
%=======================================================================
%\begin{listadeabreviaturas}{seg., segs.}
%\item[ampl.] ampliado, -a
%\item[atual.] atualizado, -a
%\item[coord.] coordenador
%\item[N.~T.] Novo Testamento
%\item[seg., segs.] seguinte, -s
%\end{listadeabreviaturas}


%=======================================================================
% Lista de Siglas (opcional).
%
% Deve ser passada como parâmetro a maior das siglas utilizadas.
%=======================================================================
%\begin{listadesiglas}{FAPERGS}
%\item[ABNT] Associação Brasileira de Normas Técnicas
%\item[CAPES] Coordenação de Aperfeiçoamento de Pessoal de Nível Superior
%\item[FAPERGS] Fundação de Amparo à Pesquisa do Estado do Rio Grande do Sul
%\end{listadesiglas}


%=======================================================================
% Lista de Símbolos (opcional).
%
% Deve ser passado o maior (mais largo) dos símbolos utilizados.
%=======================================================================
%\begin{listadesimbolos}{Ca}
%\item[\textsuperscript{o}C] Graus Celsius
%\item[Al] Alumínio
%\item[Ca] Cálcio
%\end{listadesimbolos}


%=======================================================================
% Sumário
%=======================================================================
\tableofcontents


%=======================================================================
% Introdução
%=======================================================================
\chapter{Introdução}

% as epígrafes nos capítulos são opcionais
\epigrafecap{
	Uma citação bacana.}
{Alguém}

	Inserir introdução\ldots


	\section{Internet das Coisas}
	
		Contextualizar e explicar o conceito de Internet das Coisas,
		porque é importante e o impacto social\ldots
	

	\section{Gerência de Redes}
	
		Contextualizar e explicar o conceito de Gerência de Redes, seu 
		histórico
		e motivações, por exemplo.


\section{O Problema}

	Expor o problema da gestão dos objetos inteligentes e como a grande 
	quantidade
	de dispositivos vai deixar este problema ainda pior.
		
	Descrever os objetivos brevemente de maneira textual e depois colocá-los em
	forma de topicos\ldots


	\subsection{Objetivo Geral}
	
		Definir, projetar e desenvolver um sistema de gerencia de dispositivos 
		inteligentes utilizando-se de modelos e protocolos da gerência de 
		redes 
		tradicional aplicado a Internet das Coisas.
	
	
	\subsection{Objetivos Específicos}
	
		\begin{itemize}
			\item 
				Aprofundar os conhecimentos sobre protocolos de Gerência de 
				Redes 
				e sobre protocolos utilizados por dispositivos aplicados à 
				Internet das Coisas;
				
			\item
				Definir e projetar um conjunto de extensões voltadas à gestão 
				de dispositivos inteligentes de forma a melhor adaptar o 
				protocolo escolhido à proposta;
				
			\item
				Definir e projetar um software de Gerência de Redes com 
				suporte 
				as extensões propostas.
				
		\end{itemize}


% FICA MAIS PARA O FINAL PARA DESCREVER OS PROXIMOS PASSOS NO DESENVOLVIMENTO
% DO TRABALHO...
\chapter{Metodologia}

	Um estudo abrangendo os principais protocolos de gerencia de redes e de 
	comunicação entre dispositivos inteligentes será realizado como ponto de 
	partida para o trabalho proposto. Este estudo tem por objetivo fazer um 
	levantamento dos protocolos de maior destaque em seu ramo, possibilitando 
	a 
	criação de um panorama contendo as principais tecnologias.
	
	De posse destes dados será feita uma avaliação para decidir quais 
	protocolos 
	apresentam as melhores características para atingir os objetivos deste 
	trabalho. Esta etapa visa selecionar quais tecnologias serão utilizadas no 
	restante do desenvolvimento do trabalho, de forma a obter a maior 
	contribuição 
	científica possível.
	
	Nesta etapa ocorrerá uma análise das funcionalidades do protocolo de 
	gerencia 
	de redes escolhido visando encontrar quais delas são úteis para o 
	paradigma de 
	Internet das Coisas e, adicionalmente, serão sugeridas melhorias 
	específicas 
	para este domínio de aplicação.
	
	Uma vez definido o conjunto de extensões necessárias ao protocolo de 
	gerencia 
	de redes selecionado, será projetado um software capaz de utilizar-se 
	destas 
	melhorias. Nesta etapa será criada a arquitetura do sistema de gerencia de 
	redes específico para a gestão de dispositivos inteligentes.
	
	Finalmente serão efetivamente desenvolvidas as melhorias propostas. Nesta 
	etapa serão implementadas as melhorias ao protocolo de gerencia de redes e 
	será também desenvolvido o sistema de gerencia de redes aplicado a gestão 
	de 
	dispositivos inteligentes no contexto de Internet das Coisas.

% ABORDAR A LITERATURA CLASSICA EM UM MOMENTO, EXPLICANDO OS PRINCIPAIS CARAS 
% E AS DEFINIÇÕES.

% DEPOIS ABORDAR A PARTE DE ARTIGOS COM TRABALHOS RELACIONADOS O QUE FIZERAM
% RELACIONANDO IoT COM GERENCIA DE REDES


\chapter{Protocolos de Gerência de Redes}

	Contextualizar o leitor com a utilidade e os objetivos de um protocolo
	de gerência de redes. Porque foram feitos, quais problemas resolvem,
	onde funcionam bem e onde não são tão bons (segurança, por exemplo).
	FCAPS\ldots
	
	Apresentar os protocolos principais, seus pontos fortes e fracos.
	
	
	
	\section{Modelo OSI (CMIP / CMOT)}

		Apresentar este protocolo
		
		
		
	\section{SNMP}

		Principal objeto de estudo do trabalho até o momento.
		Provavelmente será o protocolo escolhido.
		
		
	
	\section{WSDM}
	
	
	\section{Management By Delegation}
	
	
	\section{AgentX}
	
	
	\section{Outros?!}
	
	
	\section{Conclusão}
	
		Apresentar o melhor candidato pra receber as melhorias e justificar
		a escolha\ldots



\chapter{Proposta para IoT}

	Abordar neste capítulo as necessidades específicas de IoT e suas 
	características.
	
	Apresentar as dificuldades (problemas) que levarão para as propostas
	de como resolvê-las com as extensões ao protocolo.
	
	
	\section{Multicast}
		
		Propor o uso de multicast para enviar diversos comandos aos objetos
		disponíveis na rede.
		
		
	\section{Tageamento de Objetos}
		
		Propor um sistema para tagear objetos com base em características
		arbitrárias definidas pelo fabricante e pelo usuário.
		
		Quero enviar este comando para todos os objetos elétricos, para
		todos que estão na cozinha, para todos os sensores...
		
		
	\section{Compressão de Headers}
		
		Conforme feito por \cite{Choi2009}
		
		
	\section{Compressão de Dados}
		
		No caso do SNMP compressão das PDUs conforme \cite{Choi2009}
		
		
	% DESPRIORIZAR PORQUE É MUITO IGUAL E IMPORTANTE NO OUTRO ARTIGO
	\section{Requests Periódicos}
		
		Configurar uma mensagem que faz com que os clientes enviem
		dados de forma periódica para o server, evitando a necessidade
		de realizar pooling para obter informações.
		\cite{Choi2009}
		
		
%cronograma revisado
%metodologia revisada
%considerações finais
\chapter{Planejamento da Implementação}
	
	Documentar o processo de implementação\ldots
	
	\section{Proxy Forwarder}
		
		Para implementar essas alterações é possível que seja necessária
		a criação de um forwarder que vai traduzir os pacotes.
		
		A melhor solução pra isso seria fazer com que o próprio cliente
		reconhecesse o novo formato, mas perderia compatibilidade com
		o padrão.
		
		Carece de maior análise e melhor detalhamento para entender 
		melhor se realmente vai ser necessário, entretanto esta estratégia
		com os Proxies foi a usada por \cite{Choi2009}.
		%TODO Avaliar quem cita o Choi
	
	
	
	
%=======================================================================
% Referências
%=======================================================================
\bibliography{aubin}


%=======================================================================
% Exemplo de Apêndice
% O Apêndice é utilizado para apresentar material complementar elaborado
% pelo próprio autor.  Deve seguir as mesmas regras de formatação do
% corpo principal do documento.
%=======================================================================
%\appendix
%\chapter{Informações Complementares}
%
%O Apêndice é utilizado para apresentar material complementar elaborado
%pelo próprio autor.  Deve seguir as mesmas regras de formatação do
%corpo principal do documento.


%=======================================================================
% Exemplo de Anexo
% O Anexo é utilizado para a ``inclusão de materiais não elaborados pelo
% próprio autor, como cópias de artigos, manuais, folders, balancetes, etc.
% e não precisam estar em conformidade com o modelo''.
%=======================================================================
%\annex
%\chapter{Artigos Publicados}
%
%O Anexo é utilizado para a ``inclusão de materiais não elaborados pelo
%próprio autor, como cópias de artigos, manuais, folders, balancetes, etc.
%e não precisam estar em conformidade com o modelo''.


\end{document}

